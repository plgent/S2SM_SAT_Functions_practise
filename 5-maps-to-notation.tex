\question % Mathematics SL - Third Edition, RS 2C, Q10.
Let $f$ and $g$ be functions defined as $f : x \mapsto 5x - 2$ and $\displaystyle{g : x \mapsto \frac{3x}{4}}$.

\begin{parts}

\part[3]
Find $f^{-1}(x)$ and $g^{-1}(x)$.

\begin{EnvFullwidth}
\begin{solutionorgrid}[2.5in]
We have
\begin{align*}
    f : y &= 5x - 2, & f^{-1} : x &= 5y - 2 \\
    & & y &= \frac{x + 2}{5}.
\end{align*}
and
\begin{align*}
    g : y &= \frac{3x}{4}, & g^{-1} : x &= \frac{3y}{4} \\
    & & y &= \frac{4x}{3}.
\end{align*}
Thus, $f^{-1}(x) = \frac{x + 2}{5}$ and $g^{-1}(x) = \frac{4x}{3}$.
\end{solutionorgrid}
\end{EnvFullwidth}

\part[1]
Find $(g \circ f)(x)$.

\begin{EnvFullwidth}
\begin{solutionorgrid}[1.5in]
We have
\begin{align*}
    g(5x - 2) &= \frac{3(5x - 2)}{4} \\
    &= \frac{15x - 6}{4}.
\end{align*}
\end{solutionorgrid}
\end{EnvFullwidth}

\part[3]
Hence, show that $(f^{-1} \circ g^{-1})(x) = (g \circ f)^{-1}(x)$.

\begin{EnvFullwidth}
\begin{solutionorgrid}[2.5in]
The first composition is
\begin{align*}
    f^{-1}(g^{-1}(x)) &= f^{-1}(\tfrac{4x}{3}) \\
    &= \frac{\frac{4x}{3} + 2}{5} \\
    &= \frac{4x + 6}{15}.
\end{align*}
By part (b)
\begin{align*}
    g \circ f : y &= \frac{15x - 6}{4}, & (g \circ f)^{-1} : x &= \frac{15y - 6}{4} \\
    & & y &= \frac{4x + 6}{15}.
\end{align*}
Thus, $(f^{-1} \circ g^{-1})(x) = (g \circ f)^{-1}(x)$.
\end{solutionorgrid}
\end{EnvFullwidth}

\end{parts}
