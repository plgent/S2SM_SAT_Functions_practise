\question % Topic 3 - Functions and sketching graphs, Q7.
Let $f$ and $g$ be functions defined as $f(x) = x^2 - 4x + 3$ and $g(x) = \abs{f(\abs{x})}$, respectively.

\begin{parts}

\part[2]
Find $g(x)$.

\begin{EnvFullwidth}
\begin{solutionorgrid}[1in]
We have
\[
    g(x) = \abs{\abs{x}^2 - 4\abs{x} + 3}.
\]
\end{solutionorgrid}
\end{EnvFullwidth}

\part[4]
Graph and clearly label both functions on the axes in Figure~\ref{fig:graph-of-absolute-value-twice}

\begin{figure}[h]
    \centering
    \begin{tikzpicture}
    \begin{axis}[
        my axis style,
        axis equal,
        ymin=-5,
        ymax=5,
        xmin=-5,
        xmax=5,
        minor tick num=1,
        grid=both,
    ]
    \ifprintanswers
    \addplot[
        domain=-5:5,
        restrict y to domain=-5:5,
        thick,
        red,
        <->,
    ] {x^2 - 4*x + 3} node[pos=.25, right] {$y = f(x)$};
    \addplot[
        domain=-5:5,
        restrict y to domain=-5:5,
        densely dashed,
        blue,
        <->,
    ] {abs(abs(x)^2 - 4*abs(x) + 3)} node[at start, left] {$y = \abs{f(\abs{x})}$};
    \fill[red] (2,-1) node[below] {$(2, -1)$} circle (2pt);
    \fill[blue] (-2,1) node[above] {$(-2, 1)$} circle (2pt);
    \draw[red] (4,-3) node[anchor=west, text width=2in, rounded corners=5pt, fill=white, draw=red] {always include interesting points, $y = \textrm{such-and-such}$, arrowheads or closed endpoints};
    \else
    \fi
    \end{axis}
    \end{tikzpicture}
    \caption{Sketch the graphs of $y = f(x)$ and $y = g(x)$ here.}
    \label{fig:graph-of-absolute-value-twice}
\end{figure}

\part[1]
Explain the graphical effect of the ``inner'' absolute value function in the definition of $g$.

\begin{EnvFullwidth}
\begin{solutionorgrid}[1in]
The graph for $x < 0$ is discarded and the graph for $x \geq 0$ is reflected in the $y$-axis.
\end{solutionorgrid}
\end{EnvFullwidth}

\part[1]
Explain the graphical effect of the ``outer'' absolute value function in the definition of $g$.

\begin{EnvFullwidth}
\begin{solutionorgrid}[1in]
The graph for $f(x) < 0$ is reflected in the $x$-axis, discarding what was there.
\end{solutionorgrid}
\end{EnvFullwidth}

\end{parts}
