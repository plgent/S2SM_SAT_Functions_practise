\question % Sample Questions 2, Q14.
Let $g$ be a function defined as $g(x) = 2\tan(x)$, for $-\pi \leq x \leq \pi$ where $\displaystyle{x \neq \pm \frac{\pi}{2}}$.

\begin{parts}

\part[2]
Explain why $g$ is a function but doesn't have an inverse function.

\begin{EnvFullwidth}
\begin{solutionorgrid}[1.5in]
As $y = g(x)$ passes the vertical line test, $g$ is a function. However, $y = g(x)$ doesn't pass the horizontal line test, so $g^{-1}$ doesn't exist.
\end{solutionorgrid}
\end{EnvFullwidth}

\uplevel{Let $f$ be a function defined as $f(x) = 2\tan(x)$, for $\displaystyle{-\frac{\pi}{2} < x < \frac{\pi}{2}}$.}

\part[1]
Explain why $f$ does have an inverse function.

\begin{EnvFullwidth}
\begin{solutionorgrid}[1in]
The function $f$ is one-to-one.
\end{solutionorgrid}
\end{EnvFullwidth}

\part[2]
Show that $\displaystyle{f^{-1}(x) = \arctan\!\pfrac{x}{2}}$.

\begin{EnvFullwidth}
\begin{solutionorgrid}[2.5in]
There is
\begin{align*}
    f : y &= 2\tan(x), & f^{-1} : x &= 2\tan(y) \\
    & & y &= \arctan\! \pfrac{x}{2}.
\end{align*}
\end{solutionorgrid}
\end{EnvFullwidth}

\part[2]
State the domain and range of $f^{-1}(x)$ in \emph{exact} form.

\begin{EnvFullwidth}
\begin{solutionorgrid}[2in]
Since
\[
    \dom(f) = \{x \in \R : -\frac{\pi}{2} < x <  \frac{\pi}{2}\}, \qquad \ran(f) = \R,
\]
there must be
\[
    \dom(f^{-1}) = \R, \qquad \ran(f^{-1}) = \{x \in \R : -\frac{\pi}{2} < x <  \frac{\pi}{2}\}.
\]
\end{solutionorgrid}
\end{EnvFullwidth}

\part[2]
On the axes in Figure~\ref{fig:graph-of-an-inverse-tangent-function} sketch the graph of $y = f^{-1}(x)$.

\ifprintanswers
\begin{figure}
    \centering
    \begin{tikzpicture}
    \begin{axis}[
        my axis style,
        xmin=-10.25,
        xmax=10.25,
        ymin=-1.75,
        ymax=1.75,
        ytick={-pi/2,-pi/4,pi/4,pi/2},
        yticklabels={$-\frac{\pi}{2}$,$-\frac{\pi}{4}$,$\frac{\pi}{4}$,$\frac{\pi}{2}$},
    ]
    \coordinate (O) at (0,0);
    \addplot[
        domain=-10:10,
        red,
        <->,
    ] {atan(x/2)} node[at end, below] {$y = f(x)$};
    \draw[semithick, red, dashed, <->] (-10,pi/2) -- (10,pi/2) node[near start, above] {$y = \frac{\pi}{2}$};
    \draw[semithick, red, dashed, <->] (-10,-pi/2) -- (10,-pi/2) node[near start, below] {$y = -\frac{\pi}{2}$};
    \fill[red] (O) circle (2pt) node[below right] {$(0, 0)$};
    \draw[red] (7.5,-.875) node[anchor=west, text width=2in, rounded corners=5pt, fill=white, draw=red] {always include dashed, labelled asymptotes, interesting points, $y = \textrm{such-and-such}$, arrowheads or closed endpoints};
    \end{axis}
    \end{tikzpicture}
    \caption{Sketch the inverse function here.}
    \label{fig:graph-of-an-inverse-tangent-function}
\end{figure}
\else
\begin{figure}[h]
    \centering
    \begin{tikzpicture}
    \begin{axis}[
        my axis style,
        height=3.5in,
        xmin=-10.25,
        xmax=10.25,
        ymin=-1.75,
        ymax=1.75,
        xtick=\empty,
        ytick=\empty,
    ]
    \addplot[
        domain=-10:10,
        red,
        <->,
        opacity=0,
    ] {atan(x/2)};
    \end{axis}
    \end{tikzpicture}
    \caption{Sketch the inverse function here.}
    \label{fig:graph-of-an-inverse-tangent-function}
\end{figure}
\fi

\part[4]
If $\displaystyle{y = \arctan \pfrac{x}{2}}$, then $\displaystyle{\frac{x}{2} = \tan(y)}$. Show by implicit differentiation that $\displaystyle{\diff{y}{x} = \frac{2}{4 + x^2}}$.

\begin{EnvFullwidth}
\begin{solutionorgrid}[4in]
We have
\begin{align*}
    \diff{}{x}(\tan(y)) &= \diff{}{x} \pfrac{x}{2} \\
    \sec^2(y) \diff{y}{x} &= \frac{1}{2} \\
    (1 + \tan^2(y))\diff{y}{x} &= \frac{1}{2} && (\textrm{Pythagoras}) \\
    \diff{y}{x} &= \frac{1}{2(1 + \tan^2(y))}.
\end{align*}
Since $\displaystyle{\tan^2(y) = \frac{x^2}{4}}$ we get
\begin{align*}
    \diff{y}{x} &= \frac{1}{2 + \frac{x^2}{2}} {\color{red}\times \frac{2}{2}} \\
    &= \frac{2}{4 + x^2}.
\end{align*}
\end{solutionorgrid}
\end{EnvFullwidth}

\end{parts}
