\question % 2017 Exam, Q15.
Consider the equation $x^3 + x^2 + x + 1 = 0$.

\begin{parts}

\part[1]
Show that $x = -1$ is a root of this equation.

\begin{EnvFullwidth}
\begin{solutionorgrid}[1.5in]
We have
\[
    (-1)^3 + (-1)^2 + (-1) + 1 = 0.
\]
\end{solutionorgrid}
\end{EnvFullwidth}

\part[1]
Show that there are no other real roots.

\begin{EnvFullwidth}
\begin{solutionorgrid}[1.5in]
On synthetic division there is
\[
    \polyhornerscheme[x = -1]{x^3 + x^2 + x + 1}
\]
The quotient is $x^2 + 1$ which has no real roots.
\end{solutionorgrid}
\end{EnvFullwidth}

\ifprintanswers
\else
\newpage
\fi
\uplevel{Let $g(x) = \sqrt{x}$ and $h(x) = x^3 + x^2 + x + 1$.}

\part[1]
Find the composite function $g(h(x))$.

\begin{EnvFullwidth}
\begin{solutionorgrid}[1in]
We have
\[
    g(h(x)) = \sqrt{x^3 + x^2 + x + 1}.
\]
\end{solutionorgrid}
\end{EnvFullwidth}

\part[1]
State the domain of $g(h(x))$.

\begin{EnvFullwidth}
\begin{solutionorgrid}[1in]
We have
\[
    \dom(g(h(x))) = \{x \in \R : x \geq -1\}.
\]
\end{solutionorgrid}
\end{EnvFullwidth}

\uplevel{Consider $f(x) = \sqrt{x^3 + x^2 + x + 1}$.}

\part[2]
Sketch the graph of $y = f(x)$ on the axes in Figure~\ref{fig:graph-of-composite-function}.

\begin{figure}[h]
    \centering
    \begin{tikzpicture}
    \begin{axis}[
        height=3.5in,
        my axis style,
        axis equal image,
        xmin=-3.5, xmax=3.5,
        ymin=-3.5, ymax=3.5,
        minor tick num=1,
        grid=both,
    ]
    \coordinate (A) at (-1,0);
    \coordinate (B) at (0,1);
    \coordinate (C) at (1,2);
    \ifprintanswers
    \addplot[
        domain=-1:1.5,
        red,
        ->,
    ] {(x^3 + x^2 + x + 1)^(1/2)} node[at end, right] {$y = f(x)$};
    \fill[red] (A) circle (2pt) node[above left] {$(-1, 0)$} (B) circle (2pt) node[below right] {$(0, 1)$} (C) circle (2pt) node[below right] {$(1, 2)$};
    \draw[red] (2,-2) node[anchor=west, text width=2in, rounded corners=5pt, fill=white, draw=red] {always include interesting points, $y = \textrm{such-and-such}$, arrowheads or closed endpoints};
    \else
    \addplot[
        domain=-1:1.5,
        red,
        ->,
        opacity=0,
    ] {(x^3 + x^2 + x + 1)^(1/2)};
    \fi
    \end{axis}
    \end{tikzpicture}
    \caption{Sketch the graph of the composite function here.}
    \label{fig:graph-of-composite-function}
\end{figure}

\part[1]
Explain why $f(x)$ has an inverse $f^{-1}(x)$.

\begin{EnvFullwidth}
\begin{solutionorgrid}[1in]
Since $f$ is a one-to-one function, it has an inverse.
\end{solutionorgrid}
\end{EnvFullwidth}

\ifprintanswers
\else
\newpage
\fi
\part[2]
Using your answer to part (e), sketch the graph of $y = f^{-1}(x)$ on the axes in Figure~\ref{fig:graph-of-inverse-of-composite-function}.

\begin{figure}[h]
    \centering
    \begin{tikzpicture}
    \begin{axis}[
        height=3.5in,
        my axis style,
        axis equal image,
        xmin=-3.5, xmax=3.5,
        ymin=-3.5, ymax=3.5,
        minor tick num=1,
        grid=both,
    ]
    \coordinate (A) at (-1,0);
    \coordinate (B) at (0,1);
    \coordinate (C) at (1,2);
    \coordinate (D) at (0,-1);
    \coordinate (E) at (1,0);
    \coordinate (F) at (2,1);
    \addplot[
        domain=-3:3,
        semithick,
        dashed,
        <->,
    ] {x} node[at end, right] {$y = x$};
    \ifprintanswers
    \addplot[
        domain=-1:1.5,
        red,
        ->,
    ] ({x^3 + x^2 + x + 1)^(1/2)}, {x}) node[at end, right] {$y = f^{-1}(x)$};
    \addplot[
        domain=-1:1.5,
        gray,
        dashed,
        ->,
    ] {(x^3 + x^2 + x + 1)^(1/2)} node[pos=1, above] {$y = f(x)$};
    \fill[gray] (A) circle (2pt) node[above left] {$(-1, 0)$} (B) circle (2pt) node[above left] {$(0, 1)$} (C) circle (2pt) node[above left, xshift=4pt] {$(1, 2)$};
    \fill[red] (D) circle (2pt) node[below right] {$(0, -1)$} (E) circle (2pt) node[above right, xshift=1pt, yshift=-5pt] {$(1, 0)$} (F) circle (2pt) node[below right] {$(2, 1)$};
    \draw[red] (2,-2) node[anchor=west, text width=2in, rounded corners=5pt, fill=white, draw=red] {always include interesting points, $y = \textrm{such-and-such}$, arrowheads or closed endpoints};
    \else
    \addplot[
        domain=-1:1.5,
        red,
        ->,
        opacity=0,
    ] {(x^3 + x^2 + x + 1)^(1/2)};
    \fi
    \end{axis}
    \end{tikzpicture}
    \caption{Sketch the graph of the inverse function here.}
    \label{fig:graph-of-inverse-of-composite-function}
\end{figure}

\part[1]
Find $f(1)$.

\begin{EnvFullwidth}
\begin{solutionorgrid}[.75in]
We have $f(1) = \sqrt{4} = 2$.
\end{solutionorgrid}
\end{EnvFullwidth}

\part[1]
Find $f^{-1}(2)$.

\begin{EnvFullwidth}
\begin{solutionorgrid}[.75in]
Since $f(1) = 2$ there is $f^{-1}(2) = 1$.
\end{solutionorgrid}
\end{EnvFullwidth}

\part[1]
If $y = f^{-1}(x)$, then $x = f(y)$. Use implicit differentiation to show that
\[
    \diff{}{x}(f^{-1}(x)) = \diff{y}{x} = \frac{1}{f'(y)}.
\]

\begin{EnvFullwidth}
\begin{solutionorgrid}[2.25in]
By the chain rule
\begin{align*}
    \diff{}{x}(x) &= \diff{}{x}(f(y)) \\
    1 &= f'(y) \diff{y}{x} \\
    \diff{y}{x} &= \frac{1}{f'(y)}.
\end{align*}
\end{solutionorgrid}
\end{EnvFullwidth}

\part[3]
Hence, find $\displaystyle{\diff{}{x}(f^{-1}(x))}$ at $x = 2$.

\begin{EnvFullwidth}
\begin{solutionorgrid}[3in]
By part (j)
\begin{align*}
    \diff{}{x}(f^{-1}(x))\Big|_{x = 2} &= \frac{1}{f'(f^{-1}(2))} \\
    &= \frac{1}{f'(1)} && (\textrm{by (i)}).
\end{align*}
Now,
\begin{align*}
    f'(x) &= \frac{1}{2}(x^3 + x^2 + x + 1)^{-\sfrac{1}{2}} \times (3x^2 + 2x + 1) \\
    &= \frac{3x^2 + 2x + 1}{2\sqrt{x^3 + x^2 + x + 1}}.
\end{align*}
Thus,
\begin{align*}
    f'(1) &= \frac{3(1) + 2(1) + 1}{2\sqrt{1 + 1 + 1 + 1}} \\
    &= \frac{3}{2}.
\end{align*}
So
\[
    \diff{}{x}(f^{-1}(x))\Big|_{x = 2} = \frac{2}{3}.
\]
\end{solutionorgrid}
\end{EnvFullwidth}

\end{parts}
