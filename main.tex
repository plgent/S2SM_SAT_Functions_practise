\documentclass[
    a4paper,
    12pt,
    addpoints,
    % answers,
    noanswers,
    hidelinks,
]
{exam}

%%% Packages and macros

\usepackage{1-preamble}

%%% Page layout

%%% Width

\extrawidth{.5in}

%%% Header and footer (cover pages)

\coverextraheadheight{2in}

\coverheader{}{\bfseries\large Test Supervision Instructions}{}
\coverfooter{}{\thepage}{}

%%% Header and footer (main)

\headrule
\lhead[\bfseries Name:\\Supervisor:]{\bfseries OAC \subject}
\chead{}
\rhead[\bfseries \papername\\\paperrules]{\iflastpage{\bfseries Extra Space \& Mathematical Formul\ae}{\bfseries \papernameshort}}
\lfoot{\scriptsize $\mathfrak{P}\mathfrak{L}\mathfrak{G}$\\\RNum{\year}}
\cfoot{\iflastpage{Page \thepage\ of \pageref{LastPage}\\End of paper}{Page \thepage\ of \pageref{LastPage}}}
\rfoot{\iflastpage{}{Score: \makebox[.375in]{\hrulefill} /\pointsonpage{\thepage}}}

%%% Points

\boxedpoints
\pointsinrightmargin
     
%%% Solutions

\SolutionEmphasis{
	\small
}       
\shadedsolutions
\definecolor{SolutionColor}{rgb}{0.8,0.9,1}


%%% Subject, paper and conditions

\newcommand{\subject}{Stage 2 Specialist Mathematics}
\newcommand{\papername}{SAT: Functions (Practise)}
\newcommand{\papernameshort}{Functions (Practise)}
\newcommand{\timelimit}{$\infty$~Minutes}
\newcommand{\conditions}{Notes, Calculator}
\newcommand{\paperrules}{\timelimit, \conditions}

\begin{document}

% \begin{coverpages}

\noindent \today

\vspace{.375in}

\noindent Dear test supervisor:

\vspace{.125in}

\noindent Thanks for supporting student learning at OAC! Please remind students to use a blue or black pen throughout and to draw graphs accurately in pencil. Students should show all their working.

\vspace{.125in}

\noindent Make sure that you both \textbf{write your names} in the space provided on Page~1, to indicate that these exam conditions were satisfied:
\begin{itemize}
    \item Time limit: \textbf{\MakeLowercase \timelimit}.
    \item Restrictions: \textbf{\MakeLowercase \conditions\ allowed}.
	\item Students shouldn't use the internet.
	\item Students shouldn't discuss their work with anyone.
\end{itemize}

\vspace{.125in}

\noindent This page can be removed before you distribute the test to students. Scan and email the completed test back to me by the due date. If the extra space wasn't used, it needn't be returned.

\vspace{.125in}

\noindent Sincerely,

\vspace{.75in}

\noindent Peter Gent \\
Open Access College

\end{coverpages}



\begin{questions}

\question % Topic 3 - Functions and sketching graphs, Q2.
Let $x > -2$ be a real number and let $f(x) = \log_{10}(x + 2)$.

\begin{parts}

\part[2]
Find $f^{-1}(x)$.

\begin{EnvFullwidth}
\begin{solutionorgrid}[2in]
Since
\[
    f : y = \log_{10}(x + 2)
\]
it follows that
\begin{align*}
	f^{-1} : x &= \log_{10}(y + 2) \\
	10^x &= 10^{\log_{10}(y + 2)} \\
	y &= 10^x - 2.
\end{align*}
Thus, $f^{-1}(x) = 10^x - 2$.
\end{solutionorgrid}
\end{EnvFullwidth}

\part[3]
Hence, show that $(f^{-1} \circ f)(x) = (f \circ f^{-1})(x) = x$.

\begin{EnvFullwidth}
\begin{solutionorgrid}[2.5in]
The first composition $(f^{-1} \circ f)(x)$ is
\begin{align*}
	f^{-1}(f(x)) &= 10^{\log_{10}(x + 2)} - 2 \\
	&= (x + 2) - 2 \\
	&= x.
\end{align*}
The second is
\begin{align*}
	f(f^{-1}(x)) &= \log_{10}((10^x - 2) + 2) \\
	&= \log_{10}(10^x) \\
	&= x.
\end{align*}
\end{solutionorgrid}
\end{EnvFullwidth}

\end{parts}


\threeast

\question % Mathematics SL - Third Edition, RS 2C, Q10.
Let $f$ and $g$ be functions defined as $f : x \mapsto 5x - 2$ and $\displaystyle{g : x \mapsto \frac{3x}{4}}$.

\begin{parts}

\part[3]
Find $f^{-1}(x)$ and $g^{-1}(x)$.

\begin{EnvFullwidth}
\begin{solutionorgrid}[2.5in]
We have
\begin{align*}
    f : y &= 5x - 2, & f^{-1} : x &= 5y - 2 \\
    & & y &= \frac{x + 2}{5}.
\end{align*}
and
\begin{align*}
    g : y &= \frac{3x}{4}, & g^{-1} : x &= \frac{3y}{4} \\
    & & y &= \frac{4x}{3}.
\end{align*}
Thus, $f^{-1}(x) = \frac{x + 2}{5}$ and $g^{-1}(x) = \frac{4x}{3}$.
\end{solutionorgrid}
\end{EnvFullwidth}

\part[1]
Find $(g \circ f)(x)$.

\begin{EnvFullwidth}
\begin{solutionorgrid}[1.5in]
We have
\begin{align*}
    g(5x - 2) &= \frac{3(5x - 2)}{4} \\
    &= \frac{15x - 6}{4}.
\end{align*}
\end{solutionorgrid}
\end{EnvFullwidth}

\part[3]
Hence, show that $(f^{-1} \circ g^{-1})(x) = (g \circ f)^{-1}(x)$.

\begin{EnvFullwidth}
\begin{solutionorgrid}[2.5in]
The first composition is
\begin{align*}
    f^{-1}(g^{-1}(x)) &= f^{-1}(\tfrac{4x}{3}) \\
    &= \frac{\frac{4x}{3} + 2}{5} \\
    &= \frac{4x + 6}{15}.
\end{align*}
By part (b)
\begin{align*}
    g \circ f : y &= \frac{15x - 6}{4}, & (g \circ f)^{-1} : x &= \frac{15y - 6}{4} \\
    & & y &= \frac{4x + 6}{15}.
\end{align*}
Thus, $(f^{-1} \circ g^{-1})(x) = (g \circ f)^{-1}(x)$.
\end{solutionorgrid}
\end{EnvFullwidth}

\end{parts}


\threeast

\question % Sample Paper - Book 1, Q2.
Let $f$ be a function defined by $\displaystyle{f(x) = \frac{2x^2}{x^2 - x - 2}}$.

\begin{parts}

\part[2]
State the domain of $f$.

\begin{EnvFullwidth}
\begin{solutionorgrid}[.75in]
We have
\[
    \dom(f) = \{x \in \R : x \neq -1, x \neq 2\}.
\]
\end{solutionorgrid}
\end{EnvFullwidth}

\part[2]
Perform polynomial division on $\displaystyle{\frac{2x^2}{x^2 - x - 2}}$.

\begin{EnvFullwidth}
\begin{solutionorgrid}[1.75in]
Long division yields
\[
    \polylongdiv{2x^2}{x^2 - x - 2}
\]
So
\[
    \frac{2x^2}{x^2 - x - 2} = 2 + \frac{2x + 4}{x^2 - x - 2}.
\]
\end{solutionorgrid}
\end{EnvFullwidth}

\part[3]
Use a limit argument, and part (b), to explain why $y = 2$ is a horizontal asymptote.

\begin{EnvFullwidth}
\begin{solutionorgrid}[2in]
By part (b) we have
\[
    \lim_{x \to \infty} f(x) = \lim_{x \to \infty} \left(2 + \frac{2x + 4}{x^2 - x - 2}\right) = 2,
\]
and
\[
    \lim_{x \to -\infty} f(x) = \lim_{x \to -\infty} \left(2 + \frac{2x + 4}{x^2 - x - 2}\right) = 2.
\]
Thus, $y = 2$ is a horizontal asymptote.
\end{solutionorgrid}
\end{EnvFullwidth}

\uplevel{When graphing functions, one should always include dashed, labelled asymptotes, any ``interesting'' points and the graph should always be labelled with, say, $y = f(x)$.}

\part[5]
On the axes in Figure~\ref{fig:graph-of-quadratic-divided-by-quadratic} sketch the graph of $y = f(x)$.

\begin{figure}[h]
    \centering
    \begin{tikzpicture}
    \begin{axis}[
        my axis style,
        axis equal,
        ymin=-5.75,
        ymax=5.75,
        xmin=-6.75,
        xmax=6.75,
        minor tick num=1,
        grid=both,
    ]
    \ifprintanswers
    \addplot[
        domain=-6.75:-1.001,
        restrict y to domain=-5.75:5.75,
        red,
        <->,
    ] {(2*x^2)/(x^2 - x - 2)};
    \addplot[
        domain=-.999:1.999,
        restrict y to domain=-5.75:5.75,
        red,
        <->,
    ] {(2*x^2)/(x^2 - x - 2)};
    \addplot[
        domain=2.001:6.75,
        restrict y to domain=-5.75:5.75,
        red,
        <->,
    ] {(2*x^2)/(x^2 - x - 2)} node[near start, right] {$y = f(x)$};
    \draw[semithick, red, dashed, <->] (-1,-5.75) -- node[near start, left] {$x = -1$} (-1,5.75) (2,-5.75) -- node[near start, right] {$x = 2$} (2,5.75) (-6.75,2) -- node[near start, above] {$y = 2$} (6.75,2);
    \fill[red] (0,0) node[above right] {$(0, 0)$} circle (2pt);
    \draw[red] (6,-4) node[anchor=west, text width=2in, rounded corners=5pt, fill=white, draw=red] {always include dashed, labelled asymptotes, interesting points, $y = \textrm{such-and-such}$, arrowheads or closed endpoints};
    \else
    \fi
    \end{axis}
    \end{tikzpicture}
    \caption{Sketch the graph of $y = f(x)$ here.}
    \label{fig:graph-of-quadratic-divided-by-quadratic}
\end{figure}

\part[3]
On the axes in Figure~\ref{fig:graph-of-quadratic-divided-by-quadratic-absolute-value} sketch the graph of $y = f(\abs{x})$.

\begin{figure}[H]
    \centering
    \begin{tikzpicture}
    \begin{axis}[
        my axis style,
        axis equal,
        ymin=-5.75,
        ymax=5.75,
        xmin=-6.75,
        xmax=6.75,
        minor tick num=1,
        grid=both,
    ]
    \ifprintanswers
    \addplot[
        domain=-6.75:-2.001,
        restrict y to domain=-5.75:5.75,
        red,
        <->,
    ] {(2*abs(x)^2)/(abs(x)^2 - abs(x) - 2)};
    \addplot[
        domain=-1.999:1.999,
        restrict y to domain=-5.75:5.75,
        red,
        <->,
    ] {(2*abs(x)^2)/(abs(x)^2 - abs(x) - 2)};
    \addplot[
        domain=2.001:6.75,
        restrict y to domain=-5.75:5.75,
        red,
        <->,
    ] {(2*abs(x)^2)/(abs(x)^2 - abs(x) - 2)} node[near start, right] {$y = f(\abs{x})$};
    \draw[semithick, red, dashed, <->] (-2,-5.75) -- node[near start, left] {$x = -2$} (-2,5.75) (2,-5.75) -- node[near start, right] {$x = 2$} (2,5.75) (-6.75,2) -- node[near start, above] {$y = 2$} (6.75,2);
    \fill[red] (0,0) node[above right] {$(0, 0)$} circle (2pt);
    \draw[red] (6,-4) node[anchor=west, text width=2in, rounded corners=5pt, fill=white, draw=red] {always include dashed, labelled asymptotes, interesting points, $y = \textrm{such-and-such}$, arrowheads or closed endpoints};
    \else
    \fi
    \end{axis}
    \end{tikzpicture}
    \caption{Sketch the graph of $y = f(\abs{x})$ here.}
    \label{fig:graph-of-quadratic-divided-by-quadratic-absolute-value}
\end{figure}

\end{parts}


\ifprintanswers
\threeast
\else
\fi

\question % Topic 3 - Functions and sketching graphs, Q7.
Let $f$ and $g$ be functions defined as $f(x) = x^2 - 4x + 3$ and $g(x) = \abs{f(\abs{x})}$, respectively.

\begin{parts}

\part[2]
Find $g(x)$.

\begin{EnvFullwidth}
\begin{solutionorgrid}[1in]
We have
\[
    g(x) = \abs{\abs{x}^2 - 4\abs{x} + 3}.
\]
\end{solutionorgrid}
\end{EnvFullwidth}

\part[4]
Graph and clearly label both functions on the axes in Figure~\ref{fig:graph-of-absolute-value-twice}

\begin{figure}[h]
    \centering
    \begin{tikzpicture}
    \begin{axis}[
        my axis style,
        axis equal,
        ymin=-5,
        ymax=5,
        xmin=-5,
        xmax=5,
        minor tick num=1,
        grid=both,
    ]
    \ifprintanswers
    \addplot[
        domain=-5:5,
        restrict y to domain=-5:5,
        thick,
        red,
        <->,
    ] {x^2 - 4*x + 3} node[pos=.25, right] {$y = f(x)$};
    \addplot[
        domain=-5:5,
        restrict y to domain=-5:5,
        densely dashed,
        blue,
        <->,
    ] {abs(abs(x)^2 - 4*abs(x) + 3)} node[at start, left] {$y = \abs{f(\abs{x})}$};
    \fill[red] (2,-1) node[below] {$(2, -1)$} circle (2pt);
    \fill[blue] (-2,1) node[above] {$(-2, 1)$} circle (2pt);
    \draw[red] (4,-3) node[anchor=west, text width=2in, rounded corners=5pt, fill=white, draw=red] {always include interesting points, $y = \textrm{such-and-such}$, arrowheads or closed endpoints};
    \else
    \fi
    \end{axis}
    \end{tikzpicture}
    \caption{Sketch the graphs of $y = f(x)$ and $y = g(x)$ here.}
    \label{fig:graph-of-absolute-value-twice}
\end{figure}

\part[1]
Explain the graphical effect of the ``inner'' absolute value function in the definition of $g$.

\begin{EnvFullwidth}
\begin{solutionorgrid}[1in]
The graph for $x < 0$ is discarded and the graph for $x \geq 0$ is reflected in the $y$-axis.
\end{solutionorgrid}
\end{EnvFullwidth}

\part[1]
Explain the graphical effect of the ``outer'' absolute value function in the definition of $g$.

\begin{EnvFullwidth}
\begin{solutionorgrid}[1in]
The graph for $f(x) < 0$ is reflected in the $x$-axis, discarding what was there.
\end{solutionorgrid}
\end{EnvFullwidth}

\end{parts}


\threeast

\question % Topic 3 - Functions and sketching graphs, Q4.
Let $n$ be a natural number and define a sequence of functions by
\[
    f_0(x) = \frac{x}{x + 1}, \qquad f_{n + 1}(x) = f_0(f_n(x)).
\]
That is, $f_1(x) = f_0(f_0(x))$, $f_2(x) = f_0(f_1(x))$, $f_3(x) = f_0(f_2(x))$, and so on.

\begin{parts}

\part[2]
Find $f_n(x)$ for $n = 1$ and $n = 2$.

\begin{EnvFullwidth}
\begin{solutionorgrid}[2in]
We have
\begin{align*}
    f_1(x) &= f_0(f_0(x)) & f_2(x) &= f_0(f_1(x))\\
    &= \frac{\frac{x}{x + 1}}{\frac{x}{x + 1} + 1} & &= \frac{\frac{x}{2x + 1}}{\frac{x}{2x + 1} + 1} \\
    &= \frac{\frac{x}{\cancel{x + 1}}}{\frac{x + (x + 1)}{\cancel{x + 1}}} & &= \frac{\frac{x}{\cancel{2x + 1}}}{\frac{x + (2x + 1)}{\cancel{2x + 1}}} \\
    &= \frac{x}{2x + 1}, & &= \frac{x}{3x + 1}.
\end{align*}
\end{solutionorgrid}
\end{EnvFullwidth}

\part[5]
Prove by mathematical induction that $\displaystyle{f_n(x) = \frac{x}{(n + 1)x + 1}}$, for all natural numbers $n$.

\begin{EnvFullwidth}
\begin{solutionorgrid}[4.25in]
\begin{proof}
$P(n)$ is: $\displaystyle{f_n(x) = \frac{x}{(n + 1)x + 1}}$, for all $n \in \N$.

\textbf{Basis}: If $n = 0$, then
\[
	f_0(x) = \frac{x}{(0 + 1)x + 1} = \frac{x}{x + 1}.
\]
So, $P(0)$ is true.

\textbf{Inductive step}: If $P(k)$ is true, then $\displaystyle{f_k(x) = \frac{x}{(k + 1)x + 1}}$, for all $k \in \N$. Now,
\begin{align*}
	f_{k + 1} &= f_0(f_k(x)) \\
	&= \frac{\frac{x}{(k + 1)x + 1}}{\frac{x}{(k + 1)x + 1} + 1} && (\textrm{by hypothesis}) \\
	&= \frac{\frac{x}{\cancel{(k + 1)x + 1}}}{\frac{x + ((k + 1)x + 1)}{\cancel{(k + 1)x + 1}}} \\
	&= \frac{x}{((k + 1) + 1)x + 1}.
\end{align*}
Since $P(0)$ is true and $P(k) \implies P(k + 1)$, by the PMI $\displaystyle{f_n(x) = \frac{x}{(n + 1)x + 1}}$ for all $n \in \N$.
\end{proof}
\end{solutionorgrid}
\end{EnvFullwidth}

\end{parts}


\triast

% \uplevel{
% \begin{definition}[Odd functions]
% \label{def:odd-functions}
% Let $f$ be a function such that for every $x$ in the domain, there also exists $-x$ in the domain. Then $f$ is an \emph{odd function} if
% \[
% 	f(-x) = -f(x).
% \]
% \end{definition}
% }

\uplevel{
\begin{theorem}[Composition of an odd function with itself is odd]
\label{thm:composition-of-an-odd-function-with-itself}
Let $n$ be a positive integer and let $f$ be an odd function. Then the composition
\[
	f^n = \underbrace{f \circ f \circ \cdots \circ f}_{n \textrm{ times}}
\]
is an odd function.
\end{theorem}
}

\question[5] % https://math.stackexchange.com/questions/2538331/odd-function-composition.
Prove Theorem~\ref{thm:composition-of-an-odd-function-with-itself} using mathematical induction.

\begin{EnvFullwidth}
\begin{solutionorgrid}[4.5in]
\begin{proof}
$P(n)$ is: if $f$ is an odd function, then $f^n$ is an odd function, for all $n \in \Z^+$.

\textbf{Basis}: If $n = 1$, then $f^1 = f$. So, $P(1)$ is true.

\textbf{Inductive step}: If $P(k)$ is true, then $f^k$ is an odd function for all $k \in \Z^+$. Now,
\begin{align*}
	f^{k + 1}(-x) &= f(f^k(-x)) \\
	&= f(-f^k(x)) && (\textrm{by hypothesis}) \\
	&= -f(f^k(x)) && (f \textrm{ is odd}) \\
	&= -f^{k + 1}(x).
\end{align*}
Since $P(1)$ is true and $P(k) \implies P(k + 1)$, by the PMI $P(n)$ is true for all $n \in \Z^+$.
\end{proof}
\end{solutionorgrid}
\end{EnvFullwidth}


\threeast

\question % Sample Questions 2, Q14.
Let $g$ be a function defined as $g(x) = 2\tan(x)$, for $-\pi \leq x \leq \pi$ where $\displaystyle{x \neq \pm \frac{\pi}{2}}$.

\begin{parts}

\part[2]
Explain why $g$ is a function but doesn't have an inverse function.

\begin{EnvFullwidth}
\begin{solutionorgrid}[1.5in]
As $y = g(x)$ passes the vertical line test, $g$ is a function. However, $y = g(x)$ doesn't pass the horizontal line test, so $g^{-1}$ doesn't exist.
\end{solutionorgrid}
\end{EnvFullwidth}

\uplevel{Let $f$ be a function defined as $f(x) = 2\tan(x)$, for $\displaystyle{-\frac{\pi}{2} < x < \frac{\pi}{2}}$.}

\part[1]
Explain why $f$ does have an inverse function.

\begin{EnvFullwidth}
\begin{solutionorgrid}[1in]
The function $f$ is one-to-one.
\end{solutionorgrid}
\end{EnvFullwidth}

\part[2]
Show that $\displaystyle{f^{-1}(x) = \arctan\!\pfrac{x}{2}}$.

\begin{EnvFullwidth}
\begin{solutionorgrid}[2.5in]
There is
\begin{align*}
    f : y &= 2\tan(x), & f^{-1} : x &= 2\tan(y) \\
    & & y &= \arctan\! \pfrac{x}{2}.
\end{align*}
\end{solutionorgrid}
\end{EnvFullwidth}

\part[2]
State the domain and range of $f^{-1}(x)$ in \emph{exact} form.

\begin{EnvFullwidth}
\begin{solutionorgrid}[2in]
Since
\[
    \dom(f) = \{x \in \R : -\frac{\pi}{2} < x <  \frac{\pi}{2}\}, \qquad \ran(f) = \R,
\]
there must be
\[
    \dom(f^{-1}) = \R, \qquad \ran(f^{-1}) = \{x \in \R : -\frac{\pi}{2} < x <  \frac{\pi}{2}\}.
\]
\end{solutionorgrid}
\end{EnvFullwidth}

\part[2]
On the axes in Figure~\ref{fig:graph-of-an-inverse-tangent-function} sketch the graph of $y = f^{-1}(x)$.

\ifprintanswers
\begin{figure}
    \centering
    \begin{tikzpicture}
    \begin{axis}[
        my axis style,
        xmin=-10.25,
        xmax=10.25,
        ymin=-1.75,
        ymax=1.75,
        ytick={-pi/2,-pi/4,pi/4,pi/2},
        yticklabels={$-\frac{\pi}{2}$,$-\frac{\pi}{4}$,$\frac{\pi}{4}$,$\frac{\pi}{2}$},
    ]
    \coordinate (O) at (0,0);
    \addplot[
        domain=-10:10,
        red,
        <->,
    ] {atan(x/2)} node[at end, below] {$y = f(x)$};
    \draw[semithick, red, dashed, <->] (-10,pi/2) -- (10,pi/2) node[near start, above] {$y = \frac{\pi}{2}$};
    \draw[semithick, red, dashed, <->] (-10,-pi/2) -- (10,-pi/2) node[near start, below] {$y = -\frac{\pi}{2}$};
    \fill[red] (O) circle (2pt) node[below right] {$(0, 0)$};
    \draw[red] (7.5,-.875) node[anchor=west, text width=2in, rounded corners=5pt, fill=white, draw=red] {always include dashed, labelled asymptotes, interesting points, $y = \textrm{such-and-such}$, arrowheads or closed endpoints};
    \end{axis}
    \end{tikzpicture}
    \caption{Sketch the inverse function here.}
    \label{fig:graph-of-an-inverse-tangent-function}
\end{figure}
\else
\begin{figure}[h]
    \centering
    \begin{tikzpicture}
    \begin{axis}[
        my axis style,
        height=3.5in,
        xmin=-10.25,
        xmax=10.25,
        ymin=-1.75,
        ymax=1.75,
        xtick=\empty,
        ytick=\empty,
    ]
    \addplot[
        domain=-10:10,
        red,
        <->,
        opacity=0,
    ] {atan(x/2)};
    \end{axis}
    \end{tikzpicture}
    \caption{Sketch the inverse function here.}
    \label{fig:graph-of-an-inverse-tangent-function}
\end{figure}
\fi

\part[4]
If $\displaystyle{y = \arctan \pfrac{x}{2}}$, then $\displaystyle{\frac{x}{2} = \tan(y)}$. Show by implicit differentiation that $\displaystyle{\diff{y}{x} = \frac{2}{4 + x^2}}$.

\begin{EnvFullwidth}
\begin{solutionorgrid}[4in]
We have
\begin{align*}
    \diff{}{x}(\tan(y)) &= \diff{}{x} \pfrac{x}{2} \\
    \sec^2(y) \diff{y}{x} &= \frac{1}{2} \\
    (1 + \tan^2(y))\diff{y}{x} &= \frac{1}{2} && (\textrm{Pythagoras}) \\
    \diff{y}{x} &= \frac{1}{2(1 + \tan^2(y))}.
\end{align*}
Since $\displaystyle{\tan^2(y) = \frac{x^2}{4}}$ we get
\begin{align*}
    \diff{y}{x} &= \frac{1}{2 + \frac{x^2}{2}} {\color{red}\times \frac{2}{2}} \\
    &= \frac{2}{4 + x^2}.
\end{align*}
\end{solutionorgrid}
\end{EnvFullwidth}

\end{parts}


\threeast

\question % 2017 Exam, Q15.
Consider the equation $x^3 + x^2 + x + 1 = 0$.

\begin{parts}

\part[1]
Show that $x = -1$ is a root of this equation.

\begin{EnvFullwidth}
\begin{solutionorgrid}[1.5in]
We have
\[
    (-1)^3 + (-1)^2 + (-1) + 1 = 0.
\]
\end{solutionorgrid}
\end{EnvFullwidth}

\part[1]
Show that there are no other real roots.

\begin{EnvFullwidth}
\begin{solutionorgrid}[1.5in]
On synthetic division there is
\[
    \polyhornerscheme[x = -1]{x^3 + x^2 + x + 1}
\]
The quotient is $x^2 + 1$ which has no real roots.
\end{solutionorgrid}
\end{EnvFullwidth}

\ifprintanswers
\else
\newpage
\fi
\uplevel{Let $g(x) = \sqrt{x}$ and $h(x) = x^3 + x^2 + x + 1$.}

\part[1]
Find the composite function $g(h(x))$.

\begin{EnvFullwidth}
\begin{solutionorgrid}[1in]
We have
\[
    g(h(x)) = \sqrt{x^3 + x^2 + x + 1}.
\]
\end{solutionorgrid}
\end{EnvFullwidth}

\part[1]
State the domain of $g(h(x))$.

\begin{EnvFullwidth}
\begin{solutionorgrid}[1in]
We have
\[
    \dom(g(h(x))) = \{x \in \R : x \geq -1\}.
\]
\end{solutionorgrid}
\end{EnvFullwidth}

\uplevel{Consider $f(x) = \sqrt{x^3 + x^2 + x + 1}$.}

\part[2]
Sketch the graph of $y = f(x)$ on the axes in Figure~\ref{fig:graph-of-composite-function}.

\begin{figure}[h]
    \centering
    \begin{tikzpicture}
    \begin{axis}[
        height=3.5in,
        my axis style,
        axis equal image,
        xmin=-3.5, xmax=3.5,
        ymin=-3.5, ymax=3.5,
        minor tick num=1,
        grid=both,
    ]
    \coordinate (A) at (-1,0);
    \coordinate (B) at (0,1);
    \coordinate (C) at (1,2);
    \ifprintanswers
    \addplot[
        domain=-1:1.5,
        red,
        ->,
    ] {(x^3 + x^2 + x + 1)^(1/2)} node[at end, right] {$y = f(x)$};
    \fill[red] (A) circle (2pt) node[above left] {$(-1, 0)$} (B) circle (2pt) node[below right] {$(0, 1)$} (C) circle (2pt) node[below right] {$(1, 2)$};
    \draw[red] (2,-2) node[anchor=west, text width=2in, rounded corners=5pt, fill=white, draw=red] {always include interesting points, $y = \textrm{such-and-such}$, arrowheads or closed endpoints};
    \else
    \addplot[
        domain=-1:1.5,
        red,
        ->,
        opacity=0,
    ] {(x^3 + x^2 + x + 1)^(1/2)};
    \fi
    \end{axis}
    \end{tikzpicture}
    \caption{Sketch the graph of the composite function here.}
    \label{fig:graph-of-composite-function}
\end{figure}

\part[1]
Explain why $f(x)$ has an inverse $f^{-1}(x)$.

\begin{EnvFullwidth}
\begin{solutionorgrid}[1in]
Since $f$ is a one-to-one function, it has an inverse.
\end{solutionorgrid}
\end{EnvFullwidth}

\ifprintanswers
\else
\newpage
\fi
\part[2]
Using your answer to part (e), sketch the graph of $y = f^{-1}(x)$ on the axes in Figure~\ref{fig:graph-of-inverse-of-composite-function}.

\begin{figure}[h]
    \centering
    \begin{tikzpicture}
    \begin{axis}[
        height=3.5in,
        my axis style,
        axis equal image,
        xmin=-3.5, xmax=3.5,
        ymin=-3.5, ymax=3.5,
        minor tick num=1,
        grid=both,
    ]
    \coordinate (A) at (-1,0);
    \coordinate (B) at (0,1);
    \coordinate (C) at (1,2);
    \coordinate (D) at (0,-1);
    \coordinate (E) at (1,0);
    \coordinate (F) at (2,1);
    \addplot[
        domain=-3:3,
        semithick,
        dashed,
        <->,
    ] {x} node[at end, right] {$y = x$};
    \ifprintanswers
    \addplot[
        domain=-1:1.5,
        red,
        ->,
    ] ({x^3 + x^2 + x + 1)^(1/2)}, {x}) node[at end, right] {$y = f^{-1}(x)$};
    \addplot[
        domain=-1:1.5,
        gray,
        dashed,
        ->,
    ] {(x^3 + x^2 + x + 1)^(1/2)} node[pos=1, above] {$y = f(x)$};
    \fill[gray] (A) circle (2pt) node[above left] {$(-1, 0)$} (B) circle (2pt) node[above left] {$(0, 1)$} (C) circle (2pt) node[above left, xshift=4pt] {$(1, 2)$};
    \fill[red] (D) circle (2pt) node[below right] {$(0, -1)$} (E) circle (2pt) node[above right, xshift=1pt, yshift=-5pt] {$(1, 0)$} (F) circle (2pt) node[below right] {$(2, 1)$};
    \draw[red] (2,-2) node[anchor=west, text width=2in, rounded corners=5pt, fill=white, draw=red] {always include interesting points, $y = \textrm{such-and-such}$, arrowheads or closed endpoints};
    \else
    \addplot[
        domain=-1:1.5,
        red,
        ->,
        opacity=0,
    ] {(x^3 + x^2 + x + 1)^(1/2)};
    \fi
    \end{axis}
    \end{tikzpicture}
    \caption{Sketch the graph of the inverse function here.}
    \label{fig:graph-of-inverse-of-composite-function}
\end{figure}

\part[1]
Find $f(1)$.

\begin{EnvFullwidth}
\begin{solutionorgrid}[.75in]
We have $f(1) = \sqrt{4} = 2$.
\end{solutionorgrid}
\end{EnvFullwidth}

\part[1]
Find $f^{-1}(2)$.

\begin{EnvFullwidth}
\begin{solutionorgrid}[.75in]
Since $f(1) = 2$ there is $f^{-1}(2) = 1$.
\end{solutionorgrid}
\end{EnvFullwidth}

\part[1]
If $y = f^{-1}(x)$, then $x = f(y)$. Use implicit differentiation to show that
\[
    \diff{}{x}(f^{-1}(x)) = \diff{y}{x} = \frac{1}{f'(y)}.
\]

\begin{EnvFullwidth}
\begin{solutionorgrid}[2.25in]
By the chain rule
\begin{align*}
    \diff{}{x}(x) &= \diff{}{x}(f(y)) \\
    1 &= f'(y) \diff{y}{x} \\
    \diff{y}{x} &= \frac{1}{f'(y)}.
\end{align*}
\end{solutionorgrid}
\end{EnvFullwidth}

\part[3]
Hence, find $\displaystyle{\diff{}{x}(f^{-1}(x))}$ at $x = 2$.

\begin{EnvFullwidth}
\begin{solutionorgrid}[3in]
By part (j)
\begin{align*}
    \diff{}{x}(f^{-1}(x))\Big|_{x = 2} &= \frac{1}{f'(f^{-1}(2))} \\
    &= \frac{1}{f'(1)} && (\textrm{by (i)}).
\end{align*}
Now,
\begin{align*}
    f'(x) &= \frac{1}{2}(x^3 + x^2 + x + 1)^{-\sfrac{1}{2}} \times (3x^2 + 2x + 1) \\
    &= \frac{3x^2 + 2x + 1}{2\sqrt{x^3 + x^2 + x + 1}}.
\end{align*}
Thus,
\begin{align*}
    f'(1) &= \frac{3(1) + 2(1) + 1}{2\sqrt{1 + 1 + 1 + 1}} \\
    &= \frac{3}{2}.
\end{align*}
So
\[
    \diff{}{x}(f^{-1}(x))\Big|_{x = 2} = \frac{2}{3}.
\]
\end{solutionorgrid}
\end{EnvFullwidth}

\end{parts}


\end{questions}

\vfill

%%% Horizontal grading table

% \begin{center}
% 	\htword{$\sum$}
% 	\gradetable[h][pages]
% \end{center}

%%% Vertical grading table

\begin{center}
	\vtword{$\sum$}
	\multicolumngradetable{2}[pages]
\end{center}


\clearpage

\fillwithgrid{3.5in}

\threeast

\begin{multicols}{2}
\setlength\columnsep{1ex}
\footnotesize
\mathleft

\subsubsection*{Circular functions}
\label{sec:circular-functions}

\begin{align*}
    &\cos^2(A) + \sin^2(A) = 1 \\
    &\cos(A \pm B) = \cos(A)\cos(B) \mp \sin(A)\sin(B) \\
    &\sin(A \pm B) = \sin(A)\cos(B) \pm \sin(B)\cos(A) \\
    &\tan(A \pm B) = \tfrac{\tan(A) \pm \tan(B)}{1 \mp \tan(A)\tan(B)} \\
    &\cos(2A) = \cos^2(A) - \sin^2(A) \\
    &\sin(2A) = 2\sin(A)\cos(A) \\
    % &\tan(2A) = \tfrac{2\tan(A)}{1 - \tan^2(A)} \\
    &2\cos(A)\cos(B) = \cos(A - B) + \cos(A + B) \\
    &2\sin(A)\sin(B) = \cos(A - B) - \cos(A + B) \\
    &2\cos(A)\cos(B) = \sin(A + B) - \sin(A - B) \\
    &2\sin(A)\cos(B) = \sin(A + B) + \sin(A - B)
\end{align*}

\subsubsection*{Measurement}
\label{sec:measurement}

\begin{align*}
    &\ell = r\theta && (\textrm{arc length}) \\
    &\textrm{area} = \tfrac{1}{2} r^2\theta && (\textrm{area of sector}) \\
    &c^2 = a^2 + b^2 - 2ab\cos(\gamma) && (\textrm{law of cosines}) \\
    &\tfrac{a}{\sin(\alpha)} = \tfrac{b}{\sin(\beta)} = \tfrac{c}{\sin(\gamma)} && (\textrm{law of sines}) \\
    &\textrm{area} = \tfrac{1}{2}ab\sin(\gamma) && (\textrm{area of triangle})
\end{align*}
\begin{center}
    \begin{tikzpicture}
    \tkzDefPoint(0,0){A}
    \tkzDefShiftPoint[A](25:2.5){B}
    \tkzDefShiftPoint[A](130:2.8){C}
    \tkzDrawPolygon[semithick](A,B,C)
    \tkzLabelPoints[font=\footnotesize](A,B)
    \tkzLabelPoint[font=\footnotesize,left](C){$C$}
    \tkzLabelSegment[above](B,C){$a$}
    \tkzLabelSegment[below](C,A){$b$}
    \tkzLabelSegment[below](A,B){$c$}
    \tkzLabelAngle[pos=.3](B,A,C){$\alpha$}
    \tkzLabelAngle[pos=.7](C,B,A){$\beta$}
    \tkzLabelAngle[pos=.6](A,C,B){$\gamma$}
    \end{tikzpicture}
\end{center}

\subsubsection*{Quadratic equations}
\label{sec:quadratic-equations}

\begin{align*}
    &x = \frac{-b \pm \sqrt{b^2 - 4ac}}{2a} && (\textrm{if } ax^2 + bx + c = 0)
\end{align*}

\subsubsection*{Linear algebra}
\label{sec:linear-algebra}

\begin{align*}
    &\mat{A}^{-1} = \det(\mat{A})^{-1} \begin{pmatrix} d & -b \\ -c & a \end{pmatrix} && (\det(\mat{A}) \neq 0) \\
    &d = \frac{\abs{Ax_0 + By_0 + Cz_0 + D}}{\sqrt{A^2 + B^2 + C^2}} && (\textrm{distance to } (x_0, y_0, z_0))
\end{align*}

\subsubsection*{Integration}
\label{sec:integration}

\begin{align*}
    &\int \frac{\dif x}{\sqrt{1 - x^2}} = \arcsin(x) + c \\
    &\int -\frac{\dif x}{\sqrt{1 - x^2}} = \arccos(x) + c \\
    &\int \frac{\dif x}{1 + x^2} = \arctan(x) + c \\
    &\int u \dif v = uv - \int v \dif u && (\textrm{IBP}) \\
    &V = \pi\int_a^b y^2 \dif x && (\textrm{around } x{\textrm{-axis}}) \\
    &V = \pi\int_{y = c}^{y = d} x^2 \dif y && (\textrm{around } y{\textrm{-axis}}) \\
    &\ell = \int_a^b\sqrt{\vec{v} \cdot \vec{v}} \dif t && (\textrm{arc length})
\end{align*}

\end{multicols}


\end{document}
